\subsection{Procesos actuales de la empresa}

A partir de la información que obtuvimos mediante el proceso de elicitación podemos dividir los procesos que se llevan a cabo en la empresa en tres grupos:
\begin{enumerate}
    \item \textbf{Creación de los proyectos}:
    \begin{enumerate}
        \item Los gerentes o los PM son contactadas por potenciales clientes.
        \item Los gerentes seleccionan un PM, quien llevará adelante el proyecto.
        \item El PM asignado al proyecto define el alcance, duración y condiciones del proyecto con el cliente y elige el proveedor utilizando una base de datos de proveedores creada en Excel.
        \item Los gerentes valida el proyecto definido por el PM, negociando con él su comisión.
        \item Los gerentes y el PM confeccionan los contratos, a partir de templates en Word, con el cliente y el proveedor.
        \item Todas las partes firman los contratos en una escribania, donde un escribano certifica las firmas.
    \end{enumerate}    
    \item \textbf{Seguimiento de los proyectos}:
    \begin{enumerate}
        \item El PM lleva adelante el seguimiento de los proyectos, creando un archivo excel por cada proyecto que supervisa.   
        \item Los gerentes obtienen actualizaciones del estado de los proyectos a partir de llamados telefonicos con los PM.
        \item En caso de una cancelación o incumplimiento del proveedor, el PM busca otro poveedor.
        \item En caso en que un cliente manifiesta una disconfomidad con el PM, el gerente selecciona otro PM para llevar adelante el proyecto.
        \item Si surgen adicionales, estos serán considerados como nuevos proyectos.
    \end{enumerate}
    \item \textbf{Finalización de los proyectos}: consistente en realizar los diversos pagos y cobros.    
\end{enumerate}

\textbf{Limitaciones actuales}

Actualmente observamos en la empresa las siguientes limitaciones:
\begin{itemize}
    \item La actualización de la base de datos de proveedores es engorrosa, ya que cada miembro de DC Constructores tiene su versión de la misma.
    \item Actualmente la información sobre los proyectos esta dispersa en varios archivos, dificultando su seguimiento.
    \item Los gerentes se enteran tarde sobre los problemas en los proyectos, ya que deben comunicarse con cada PM para obtener actualizaciones o esperar su llamado.
    \item Gerentes realizan tareas tediosas que no agregan valor, tales como búsqueda de proveedores, creaciones de contratos y llamados constantes a los PM.
    \item La modalidad actual no es escalable.
    \item No hay registro sobre proyectos pasados que podrian llevar a mejoras en la selección de los proveedores y de los PM.
\end{itemize}

\subsection{Presentación sistema}
El sistema que proponemos se caracteriza por simplificar las tareas dentro de la empresa, optimizar la elección de los proveedores, agilizar las comunicaciones, y permitir controles inmediatos. Para lograr estos fines, suponemos las siguientes presunciones de dominio:

\begin{enumerate}
    \item \textbf{Estructura de la empresa}:
    \begin{enumerate}
        \item Gerente
        \item PMs
        \item Empleados que se encargan de hacer Data entry
        \item Contadores
    \end{enumerate}    
    \item \textbf{Creación de los proyectos}:
    \begin{enumerate}
        \item El cliente pide propuesta de proyecto o bien al PM, o bien al Gerente o bien al Sistema
        \item El cliente tiene acceso a Internet
        \item Todo proyecto nuevo se tiene que registrar en el sistema
        \item Cada proyecto tiene exactamente un PM asignado
        \item Hay un PM al menos en la empresa
        \item El PM es el que negocia alcance, duración y condiciones con el cliente
        \item El gerente siempre debe validar a los proveedores
        \item Todo proveedor tiene que tener el seguro de caución actualizado al día
        \item Los PM tienen una comisión por proyecto
        \item Todos los contratos deben ser certificados por una escribanía
    \end{enumerate}    
    \item \textbf{Seguimiento de los proyectos}:
    \begin{enumerate}
        \item El PM es el encargado de contactar al proveedor y saber si hay un problema
    \end{enumerate}
    \item \textbf{Finalización de los proyectos}:
    \begin{enumerate}
        \item Todo proyecto finaliza
        \item El PM es el encargado de finalizar el proyecto
        \item El contador se encarga de manejar todos los pagos cuando se termina el proyecto
    \end{enumerate}
\end{enumerate}


\subsection{Requerimientos}
\begin{enumerate}

%	\textbf{Creación de los proyectos}:
	\item Proveer interfaz de mandar propuesta
	\item Cliente manda propuesta mediante el Sistema $\rightarrow$ Notificar gerente
	\item Proveer interfaz de creación de proyecto
	\item Proveer interfaz gráfica de carga de datos de los PM
	\item Crear ranking de PM
	\item Evaluaciones cargadas en el sistema
	\item Recalcular ranking de PMs usando las evaluaciones
	\item Ordenar PMs por cantidad de proyectos actuales
	\item Proveer interfaz gráfica selección de PM
	\item Nuevo proyecto registrado en el sistema y PM asignado $\rightarrow$ Notificar gerente y PM
	\item Sistema provee interfaz de cargar y mandar propuesta
	\item Sistema provee interfaz de cargar y mandar respuesta
	\item Proveer interfaz de carga de alcance, duración y condiciones
	\item Notificar a los proveedores cuando su seguro este vencido
	\item Proveer interfaz de carga de datos de proveedores
	\item Crear ranking de proveedores
	\item Evaluaciones actualizadas en el sistema
	\item Recalcular ranking de proveedores usando las evaluaciones
	\item Aplicar filtros para proveedores (rubro, ubicación geográfica)	
	\item Proveer interfaz gráfica de consulta de disponibilidad
	\item Consultar proveedores con seguro de caución vencido
	\item Proveedores seleccionados $\rightarrow$ Sistema manda mail con formulario pidiendo presupuesto y disponibilidad para el proyecto actual
	\item Proveedor llena el formulario $\rightarrow$ Notificar respuesta del proveedor al PM	
	\item Proveer interfaz gráfica selección de proveedor
	\item Proveedores asignados $\rightarrow$ Notificar a los proveedores elegidos y mandar mail de agradecimiento a los no elegidos
	\item Proveedores asignados $\rightarrow$ pedir validación Gerente
	\item Proveer interfaz de validar proveedores
	\item Proveer interfaz de elegir precontrato
	\item Notificar PM, Cliente y Proveedores que ya está todo el proyecto validado y hay que firmar los contratos
	
%	\textbf{Seguimiento de los proyectos}:
	\item Proveedor nuevo y anterior iguales
	\item Proveer interfaz para cargar actualizaciones de proyecto
	\item Detectar si el PM no sube actualizaciones del proyecto
	\item Cambios o Problemas en proyecto $\rightarrow$ Notificar al Gerente
	\item Proveer interfaz para marcar como cancelado un proyecto
	
	
%	\textbf{Finalización de los proyectos}:
	\item Proveer interfaz para marcar como finalizado un proyecto
	\item El proyecto se marca como finalizado $\rightarrow$ Notificar al Gerente
	\item Enviar detalles y costos al Contador
	\item Proveer interfaz para evaluar al PM
	
	
\end{enumerate}
