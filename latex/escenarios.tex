Para ilustrar el funcionamiento del Sistema propuesto, detallamos a continuación como sería el flujo de trabajo en diferentes escenarios:

\subsection{Creación de proyectos}

\begin{enumerate}
    \item Un cliente se contacta con DC Construcciones requiriendo los servicios de la empresa. Esto puede pasar:
        \begin{itemize}
            \item Enviando una pedido a través del Sistema (en cuyo caso el Sistema después notifica al Gerente), o
            \item Llevando el pedido directamente a un Gerente
        \end{itemize}
    \item Los gerentes o el empleado crean un nuevo proyecto en el Sistema detallando los datos del cliente.
    \item Utilizando el Sistema para asesorarlos en su decisión, los gerentes designan al PM para el proyecto.
        La designación es cargada en el Sistema por los gerentes o el empleado.
    \item El PM asignado es notificado por el Sistema sobre el nuevo proyecto en el cual esta a cargo.
    \item El PM asignado se encarga de:
        \begin{itemize}
            \item Consensuar el alcance y detalles del proyecto con el cliente. Si este se comunicó con la empresa a través del Sistema, entonces el PM interacciona con él a través del Sistema, en caso contrario lo hace personalmente. Una vez que se llega a consenso el PM (\textbf{el empleado no, no ?}) carga el alcance y detalles en el Sistema.
            \item Elegir proveedores, asesorado por el Sistema. Para esto:
                \begin{itemize}
                    \item El Sistema propone proveedores basado en ranking y filtros. Los proveedores propuestos tienen todos el seguro de caución al día.
                    \item Luego, el PM puede notificar a través del Sistema a aquellos proveedores que tengan una cuenta en el mismo, contandoles el alcance del proyecto y pidiendo
                        presupuesto. Si el PM quisiera contactarse con un proveedor que no tiene cuenta en el Sistema debe hacerlo personalmente, refelejando luego lo acordado en el Sistema.
                    \item Los proveedores responden por el mismo medio por el cual fueron contactados.
                    \item Al llegar a un arreglo con un proveedor, el PM lo asigna al proyecto en el Sistema.
                \end{itemize}
        \end{itemize}
    \item Una vez que un proyecto tiene cargado su alcance y proveedores en el Sistema, este envía una notificación a los gerentes pidiendo su validación, la cual se lleva a cabo en el Sistema.
    \item Los gerentes validan el proyecto en el Sistema, y envían presupuesto al cliente (personalmente o a través del Sistema).
    \item Si el Cliente acepta (personalmente o a través del Sistema), los gerentes arman un pre-contrato utilizando el Sistema, que propone diferentes templates basados en las características del proyecto.
    \item Luego, los gerentes, PM, Cliente y proveedores afinan los detalles del pre-contrato en la Escribanía, dónde luego firman todos.
\end{enumerate}

\subsection{Seguimiento de proyectos}
\begin{enumerate}
    \item Una vez comenzado un proyecto, el PM asignado es el encargado de subir actualizaciones en el Sistema. Estas actualizaciones las puede subir el PM mismo, o las puede subir el Empleado.
    \item Cada proyecto puede ser configurado en el Sistema para tener actualizaciones dentro de un cierto período de tiempo.
    \item En caso de que se esté por agotar el período de tiempo y no haya una actualización, el Sistema le envía una notificación al PM pidiendolé que actualice.
    \item En caso de que se agote el período de tiempo y no haya una actualización, el Sistema envía le notifica esto a los gerentes.
    \item Además de las actualizaciones obligatorias, el PM puede subir actualizaciones en cualquier otro momento, por ejemplo para indicar que sucedió algún problema en el proyecto en cuyo caso el Sistema también notifica a los gerentes.
    \item En caso de que el proveedor quiera cancelar, puede hacerlo personalmente (y el PM lo refleja en el Sistema) o diréctamente a través del Sistema.
\end{enumerate}

\subsection{Finalización de proyectos}
\begin{enumerate}
    \item Una vez que finaliza un proyecto, el PM asignado lo marca así en el Sistema.
    \item Luego se hacen varias encuestas:
        \begin{itemize}
            \item El PM evalua a los proveedores. Esta evaluación la puede cargar en el Sistema el PM mismo o el Empleado.
            \item El Gerente evalua al PM. Esta evaluación la puede cargar en el Sistema el Gerente mismo o el Empleado.
            \item El Cliente evalua al PM. Si tiene internet lo hace diréctamente a través del Sistema, si no a a través del Empleado quien después carga la evaluación en el Sistema.
        \end{itemize}
      \item Por último, el Sistema le envía los costos y detalles del proyecto al Contador.
\end{enumerate}

\subsection{Actualización proveedores en el Sistema}
\begin{enumerate}
    \item Un proveedor puede registrarse en el Sistema directamente o mediante alguna persona de la empresa.
    \item A continuación, el proveedor debe enviar su seguro de caución al día. Devuelta, esto puede hacerlo directamente en el Sistema o mediante alguna persona de la empresa.
    \item Más adelante, aquellos proveedores que estén cargados en el Sistema y tengan un seguro de caución vencido serán notificados a través del Sistema para que lo actualicen.
      De no poder acceder al Sistema, será el Empleado el encargado de contactarse con ellos y actualizar su seguro de caución en el Sistema.
\end{enumerate}
