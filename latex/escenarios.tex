Para ilustrar el funcionamiento del Sistema propuesto, detallamos como sería el flujo de trabajo en las siguientes situaciones:
\begin{enumerate}
    \item Creación de proyectos.
    \item Seguimiento de proyectos.
    \item Finalización de proyectos.
\end{enumerate}

\subsection{Creación de proyectos}

\begin{enumerate}
    \item Un cliente se contacta con DC Construcciones requiriendo los servicios de la empresa. Esto puede pasar:
        \begin{itemize}
            \item Enviando una pedido a través del Sistema (en cuyo caso el Sistema después notifica al Gerente), o
            \item Llevando el pedido directamente a un Gerente
        \end{itemize}
    \item Los gerentes o el empleado crean un nuevo proyecto en el Sistema detallando los datos del cliente.
    \item Utilizando el Sistema para asesorarlos en su decisión, los gerentes designan al PM para el proyecto.
        La designación es cargada en el Sistema por los gerentes o el empleado.
    \item El PM asignado es notificado por el Sistema sobre el nuevo proyecto en el cual esta a cargo.
    \item El PM asignado se encarga de:
        \begin{itemize}
            \item Consensuar el alcance y detalles del proyecto con el cliente. Si este se comunicó con la empresa a través del Sistema, entonces el PM interacciona con él a través del Sistema, en caso contrario lo hace personalmente. Una vez que se llega a consenso el PM (\textbf{el empleado no, no ?}) carga el alcance y detalles en el Sistema.
            \item Elegir proveedores, asesorado por el Sistema. Para esto:
                \begin{itemize}
                    \item El Sistema propone proveedores basado en ranking y filtros.
                    \item El PM elige un proveedor de esa lista. Si dicho proveedor tiene el seguro de caución vencido, entonces el PM debe (ya sea por el Sistema o personalmente) pedir que envien devuelta dicho seguro para actualizarlo en el Sistema.
                    \item Luego, el PM se contacta con el proveedor (a través del Sistema o personalmente) para contarle el proyecto y pedir presupuesto.
                    \item El proveedor responde por el mismo medio por el cual fue contactado.
                    \item Al llegar a un arreglo, el PM asigna ese proveedor al proyecto en el Sistema.
                \end{itemize}
        \end{itemize}
    \item Una vez que un proyecto tiene cargado su alcance y proveedores en el Sistema, este envía una notificación a los gerentes pidiendo que validación, la cual se lleva a cabo en el Sistema.
    \item Los gerentes validan el proyecto en el Sistema, y envían presupuesto al cliente (personalmente o a través del Sistema).
    \item Si el Cliente acepta (personalmente o a través del Sistema), los gerentes arman un pre-contrato utilizando el Sistema, que propone diferentes templates basados en las características del proyecto.
    \item Los gerentes, PM, Cliente y proveedores afinan los detalles del pre-contrato en la Escribanía, dónde luego firman todos.
\end{enumerate}

\subsection{Seguimiento de proyectos}
\begin{enumerate}
    \item Una vez comenzado un proyecto, el PM asignado es el encargado de subir actualizaciones en el Sistema.
    \item Cada proyecto puede ser configurado en el Sistema para tener actualizaciones dentro de un cierto período de tiempo.
    \item En caso de que se esté por agotar el período de tiempo y no haya una actualización, el Sistema le envía una notificación al PM pidiendolé que actualice.
    \item En caso de que se agote el período de tiempo y no haya una actualización, el Sistema envía le notifica esto a los gerentes.
    \item Además de las actualizaciones obligatorias, el PM puede subir actualizaciones en cualquier otro momento, por ejemplo para indicar que sucedió algún problema en el proyecto en cuyo caso el Sistema también notifica a los gerentes.
    \item En caso de que el proveedor quiera cancelar, puede hacerlo personalmente (y el PM lo refleja en el Sistema) o diréctamente a través del Sistema.
\end{enumerate}

\subsection{Finalización de proyectos}
\begin{itemize}
    \item Un cliente se contacta con DC Construcciones requiriendo los servicios de la empresa.
\end{itemize}
